\documentclass[11pt]{article}
\usepackage[margin=1in]{geometry} % Adjust margins if needed
\usepackage{amsmath} % For \text and other math commands
\usepackage{amssymb}
%\usepackage{fontspec} % Removed for pdfLaTeX compatibility
%\usepackage{unicode-math} % Removed for pdfLaTeX compatibility
%\setmainfont{Latin Modern Roman} % Removed for pdfLaTeX compatibility
%\setmathfont{Latin Modern Math} % Removed for pdfLaTeX compatibility

\begin{document}
\noindent Hw3a Lab Report Ongoing

\noindent Ryan Oates

\noindent [Notation Key]

\noindent GateSymbolNotation = [List of logic gate symbols used in this design]

\noindent ; e.g., AND ($\land$), OR ($\lor$), NOT ($\lnot$), etc., to represent gate operations in Boolean expressions.

\vspace{0.5em}

\noindent WireConnectionSymbols = [Notation for wires and connections in circuit diagrams]

\noindent ; e.g., lines for connections, a dot for junction, labels/arrows for inputs and outputs.

\vspace{0.5em}

\noindent LogicLevelsRepresentation = [Definition of logic high and low levels]

\noindent ; e.g., use `1` (HIGH/True) and `0` (LOW/False) to denote binary logic levels.

\vspace{0.5em}

\noindent BooleanExpressionNotation = [Format for writing Boolean formulas]

\noindent ; e.g., use `+' for OR, `$\cdot$' or concatenation for AND, overline or `!' for NOT (A + B means A OR B, AB or A$\cdot$B means A AND B, $\overline{A}$ means NOT A).

\noindent ; Note: These notations are used throughout (in the truth table, K-map, expressions, and diagrams). See Glossary for definitions of terms and symbols.

\vspace{0.5em}

\noindent ; $\checkmark$/$\times$ Verification:

\noindent verify\_symbols\_defined = ($\checkmark$/$\times$) ; All necessary gate symbols and wire notations are defined above. (Comment: \_\_\_)

\noindent verify\_logic\_levels = ($\checkmark$/$\times$) ; Logic 0/1 representation is clear and used consistently. (Comment: \_\_\_)

\noindent verify\_notation\_consistency = ($\checkmark$/$\times$) ; Boolean expression notation is consistent across all sections. (Comment: \_\_\_)

\vspace{0.5em}

\noindent [Possible Inputs]

\noindent InputVariables = [List of input variable names]

\noindent ; e.g., A, B, C (representing all independent inputs to the logic circuit).

\vspace{0.5em}

\noindent InputDomain = [All possible values or conditions for each input]

\noindent ; e.g., each input $\in$ \{0,1\} for binary logic (enumerate any constraints or don't-care conditions if applicable).

\noindent ; Note: These inputs define the domain for the Truth Table.

\noindent ... (about 599 lines omitted) ...

\noindent \begin{tabular}{c|c|c|c|c|l}
Time  & A & B & Sum & Carry & Analysis \\
20-40ns & 1 & 0 &  1  &   0   & A transition \\
40-60ns & 1 & 1 &  0  &   1   & B transition, carry generated \\
60-80ns & 0 & 1 &  1  &   0   & A transition \\
80-100ns& 0 & 0 &  0  &   0   & Return to initial state
\end{tabular}

\vspace{0.5em}

\noindent [DESIGN\_HISTORY]

\noindent iterations="""

\noindent v1: Initial design (redundant inputs)

\noindent v2: Simplified inputs

\noindent v3: Corrected carry generation

\noindent v4: Final verified design

\noindent """

\vspace{0.5em}

\noindent [VISUALIZATION\_GUIDELINES]

\noindent setup\_steps="""

\noindent 1. Right-click signal names for color options

\noindent 2. Tools-->Settings-->Waveforms for trace width

\noindent 3. Tools-->Color preferences for background

\noindent 4. Maximize RGB values (255) for clarity

\noindent """

\vspace{0.5em}

\noindent [VERIFICATION\_CHECKLIST]

\noindent $\checkmark$ truth\_table\_verified

\noindent $\checkmark$ timing\_requirements\_met

\noindent $\checkmark$ carry\_generation\_correct

\noindent $\checkmark$ waveform\_visualization\_optimized

\noindent KMapLayout = [K-Map cell arrangement]

\noindent ; e.g., 2x2, 4x4 grid with Gray code ordering of inputs on rows/columns, as appropriate for the number of variables.

\noindent KMapFilling = [Assignment of output values to K-map cells]

\noindent; e.g., place 1s and 0s in K-map cells corresponding to truth table outputs (1 for minterms where F=1).

\noindent KMapGrouping = [Identify groups of adjacent 1s (or 0s for POS) in the K-map]

\noindent ; e.g., list groups by cells covered (binary indices) and the common variables for each group.

\noindent KMapSimplificationSteps = [Step-by-step simplification]

\noindent; e.g., "Group1 (cells 0101, 0100, 0000, 0001) yields term $\lnot$A$\cdot$$\lnot$C", etc.

\noindent KMapResultExpression = [Simplified Boolean expression from K-map]

\noindent; e.g., F = $\lnot$A$\cdot$$\lnot$C + B$\cdot$$\lnot$C (result of grouping). This is the minimized Sum-of-Products form derived via the K-map.

\noindent ; Note: K-map simplification is based on the Truth Table and yields a minimal expression. The result will be used to inform the Gate Operations and is cross-checked by Boolean algebra in [Boolean Expression Reduction].

\vspace{0.5em}

\noindent ; $\checkmark$/$\times$ Verification:

\noindent verify\_grouping\_complete = ($\checkmark$/$\times$) ; All 1-cells are covered by groups (including any don't-cares if used), with groups sized in powers of 2. (Comment: \_\_\_)

\noindent verify\_minimal\_expression = ($\checkmark$/$\times$) ; The K-map result is a minimal expression (no further reduction possible, no redundant groups). (Comment: \_\_\_)

\noindent verify\_expression\_correct = ($\checkmark$/$\times$) ; The simplified expression from K-map produces the same output as the original truth table (validated by checking key input cases or re-deriving truth table from it). (Comment: \_\_\_)

\vspace{0.5em}

\noindent [Gate Operations]

\noindent ; Define the logic gates implementing the simplified expression:

\noindent Gate1 = [Gate type and inputs -> output]

\noindent ; e.g., G1: AND gate with inputs A and B, output = X (implements X = A $\land$ B).

\noindent Gate2 = [Gate type and inputs -> output]

\noindent ; e.g., G2: NOT gate with input C, output = Y (implements Y = $\lnot$C).

\noindent Gate3 = [Gate type and inputs -> output]

\noindent ; e.g., G3: OR gate with inputs X and Y, output = F (implements F = X + Y).

\noindent GateOperationNotes = [Any additional info about gate behavior/timing]

\noindent ; e.g., All gates are assumed ideal for logic levels (output transitions occur after a small propagation delay). If timing is critical, note propagation delays or if this is a synchronous circuit, note clocking.

\noindent ; Note: The above gates realize the expression from [K-Map Reduction] (F = X + Y in the example) using standard logic symbols from [Notation Key]. Each gate's logical function (AND, OR, NOT) is defined in the Glossary (e.g., AND gate outputs 1 only if all inputs are 1).

\vspace{0.5em}

\noindent ; $\checkmark$/$\times$ Verification:

\noindent verify\_gates\_complete = ($\checkmark$/$\times$) ; All parts of the simplified Boolean expression are implemented by the listed gates (no missing term or signal). (Comment: \_\_\_)

\noindent verify\_output\_consistency = ($\checkmark$/$\times$) ; The output from the gate network matches the expected truth table output for each input combination (circuit logic is correct). (Comment: \_\_\_)

\noindent verify\_notation\_adherence = ($\checkmark$/$\times$) ; Gate symbols and wiring in the described implementation follow the Notation Key (correct symbols for each gate, proper label names). (Comment: \_\_\_)

\noindent verify\_timing\_addressed = ($\checkmark$/$\times$) ; Any necessary timing considerations (propagation delay, gate switching speed) are noted and acceptable for the design. (Comment: \_\_\_)

\vspace{0.5em}

\noindent [Circuit Diagram Representation]

\noindent DiagramIllustration = [Reference or description of the circuit diagram]

\noindent ; (In an actual document, this could be an embedded schematic image or ASCII diagram of the circuit. e.g., a drawn logic diagram with gates labeled G1, G2, G3 as above.)

\noindent DiagramNotation = [Explanation of diagram symbols/legends]

\noindent ; e.g., AND/OR/NOT gate shapes per standard, dots on lines for junctions, distinctive symbols for inputs and outputs as defined in Notation Key.

\noindent DiagramLabels = [List of labeled signals in diagram]

\noindent ; e.g., label inputs (A, B, C), outputs (F), and intermediate nodes (X, Y) corresponding to the Gate Operations.

\noindent ; Note: The circuit diagram provides a visual confirmation of the [Gate Operations]. It should use the symbols from [Notation Key] and reflect the connections described (e.g., output of G1 feeds one input of G3, etc.).

\vspace{0.5em}

\noindent ; $\checkmark$/$\times$ Verification:

\noindent verify\_diagram\_accuracy = ($\checkmark$/$\times$) ; Diagram matches the gate-level implementation (all gates and connections correspond to those listed in Gate Operations). (Comment: \_\_\_)

\noindent verify\_label\_consistency = ($\checkmark$/$\times$) ; Every signal in the diagram (inputs, outputs, nodes) is clearly labeled and matches the naming in the text/tables. (Comment: \_\_\_)

\noindent verify\_symbol\_standard = ($\checkmark$/$\times$) ; All symbols in the diagram are standard and were defined in the Notation Key (no undefined symbols or ambiguous notations). (Comment: \_\_\_)

\vspace{0.5em}

\noindent [Full Signal Analysis]

\noindent ; Analyze signal propagation and node equations in the circuit:

\noindent NodeEquations = [List Boolean equations for intermediate nodes]

\noindent ; e.g., X = A $\land$ B, Y = $\lnot$C (from Gate1 and Gate2 outputs as defined in Gate Operations).

\noindent OutputEquation = [Boolean equation for output in terms of nodes or inputs]

\noindent ; e.g., F = X + Y (which, when expanded, matches the simplified Boolean expression).

\noindent PropagationSteps = [Describe the sequence of signal propagation through the circuit]

\noindent ; e.g., "Inputs A,B feed AND gate -> X. Input C feeds NOT gate -> Y. Then X and Y feed OR gate -> F. Thus, a change in A or B propagates through G1 then G3 to F; a change in C propagates through G2 then G3."

\noindent TimingAnalysis = [If applicable, note the critical path or propagation delay]

\noindent ; e.g., "Worst-case propagation: a change on A or B travels through two gates (AND then OR) to affect F, whereas a change on C travels through NOT then OR. Assuming each gate has similar delay, the critical path is two gates long. No hazards/glitches were observed as all inputs feed combinational gates with proper synchronization (see Propagation Delay in Glossary)."

\noindent ; Note: This analysis verifies that each intermediate node's logic (node equations) and the timing of signal changes produce the correct final output as specified in the [Truth Table]. It ties the static logic design to dynamic behavior (propagation).

\vspace{0.5em}

\noindent; $\checkmark$/$\times$ Verification:

\noindent verify\_node\_logic = ($\checkmark$/$\times$) ; Every intermediate node equation is consistent with the intended logic and the overall Boolean expression. (Comment: \_\_\_)

\noindent verify\_propagation\_correctness = ($\checkmark$/$\times$) ; For each input scenario, following the signal path through nodes yields the correct output (matches truth table -- ensures no logical discrepancies at any stage). (Comment: \_\_\_)

\noindent verify\_timing\_consistency = ($\checkmark$/$\times$) ; Signal propagation times are acceptable and do not violate any design requirements (or timing not an issue for static combinational logic). Any potential glitches or race conditions checked. (Comment: \_\_\_)

\vspace{0.5em}

\noindent [SOP Reduction]

\noindent InitialSOP = [Sum-of-Products expression derived from the truth table]

\noindent ; e.g., list of minterms: F = A$\cdot$B$\cdot$$\lnot$C + A$\cdot$B$\cdot$C + ... (each term corresponds to an input combination where output=1 in the Truth Table).

\noindent ReductionSteps = [Algebraic reduction steps applied to simplify the SOP]

\noindent ; e.g., combine terms using Boolean algebra theorems:

\noindent ;   F = A$\cdot$B$\cdot$$\lnot$C + A$\cdot$B$\cdot$C  -> factor A$\cdot$B: F = A$\cdot$B($\lnot$C + C)

\noindent ;   -> simplify ($\lnot$C + C = 1): F = A$\cdot$B

\noindent ;   (show step-by-step simplification of the SOP expression).

\noindent ReducedSOP = [Final minimized SOP expression]

\noindent ; e.g., F = A$\cdot$B + $\lnot$C$\cdot$B (the simplified sum-of-products form, which should match the K-Map result).

\noindent ; Note: This section provides an algebraic simplification of the Sum-of-Products derived from the Truth Table. The final result should coincide with the expression from [K-Map Reduction], confirming the minimization is correct.

\vspace{0.5em}

\noindent ; $\checkmark$/$\times$ Verification:

\noindent verify\_all\_minterms\_used = ($\checkmark$/$\times$) ; Initial SOP includes all required minterms for output=1 (covers every truth table case where F=1). (Comment: \_\_\_)

\noindent verify\_algebraic\_steps = ($\checkmark$/$\times$) ; Boolean algebra steps are correctly applied (each simplification step is valid). (Comment: \_\_\_)

\noindent verify\_sop\_matches\_kmap = ($\checkmark$/$\times$) ; The reduced SOP expression matches the simplified result from K-Map Reduction (thus both methods agree on final expression). (Comment: \_\_\_)

\vspace{0.5em}

\noindent [Boolean Expression Reduction]

\noindent InitialExpression = [Original Boolean expression before simplification]

\noindent ; e.g., starting expression from requirements or the full SOP expression before reduction.

\noindent TargetExpression = [Target (simplified) Boolean expression]

\noindent ; e.g., the expected simplified expression (from K-Map or known simplest form).

\noindent ReductionTechnique = [Method used for reduction]

\noindent ; e.g., algebraic manipulation, applying identities (consensus theorem, De Morgan's law, etc.) to reach the simplified form.

\noindent ReductionProof = [Proof or validation of equivalence]

\noindent ; e.g., a brief statement: "Verified equivalence by truth table comparison or Boolean algebra proof showing initial expression equals simplified expression."

\noindent FinalBooleanExpression = [The final simplified Boolean expression for the design]

\noindent ; e.g., F = A$\cdot$B + $\lnot$C$\cdot$B (from our example), which is the form implemented by the circuit.

\noindent ; Note: This section ensures the logical expression of the circuit is fully simplified and correct. It should confirm that the expression derived (via K-Map or SOP) is indeed the most reduced form and is logically equivalent to the original specification. This final expression is the one realized in [Gate Operations].

\vspace{0.5em}

\noindent ; $\checkmark$/$\times$ Verification:

\noindent verify\_equivalence\_proven = ($\checkmark$/$\times$) ; Demonstrated that the initial and final expressions are equivalent (via truth table or symbolic proof). (Comment: \_\_\_)

\noindent verify\_expression\_minimal = ($\checkmark$/$\times$) ; The final Boolean expression is simplified to minimal terms/literals (no further simplification possible). (Comment: \_\_\_)

\noindent verify\_consistency\_final = ($\checkmark$/$\times$) ; The final expression aligns with the implemented circuit (the gates in the design realize this expression) and with the truth table outputs. (Comment: \_\_\_)

\vspace{0.5em}

\noindent [Direct Links]

\noindent NotationKey\_related     = Used by all sections for consistent symbols (e.g., gate symbols in Gate Operations, logic levels in Truth Table).

\noindent PossibleInputs\_related  = Feeds into Truth Table (defines all combinations for which outputs are determined).

\noindent TruthTable\_related      = Basis for K-Map Reduction and SOP Reduction (provides the raw output combinations for simplification).

\noindent KMapReduction\_related   = Simplifies the Truth Table outputs to a minimal expression (used by Gate Operations, cross-checked in Boolean Expression Reduction).

\noindent GateOperations\_related  = Implements the simplified expression (from K-Map/SOP) with physical gates, referencing symbols from Notation Key.

\noindent CircuitDiagram\_related  = Visual representation of Gate Operations (wiring and gates), using the notations defined in Notation Key.

\noindent FullSignalAnalysis\_related = Verifies signal flow and timing from inputs to output (ensures dynamic behavior matches static Truth Table logic).

\noindent SOPReduction\_related    = Alternative/parallel simplification method (derives simplified expression from truth table, should match K-Map result).

\noindent BooleanExprReduction\_related = Final verification of expression simplification (confirms K-Map and SOP results, ensuring correct implementation).

\noindent Glossary\_related        = Provides definitions of terms and symbols (e.g., gates, K-Map, SOP) used throughout the document for quick reference.

\vspace{0.5em}

\noindent [Glossary]

\noindent AND Gate = Logic gate that outputs 1 (true) only if \textbf{all} its inputs are 1. (Logical conjunction; symbol $\land$ or \&)

\noindent OR Gate  = Logic gate that outputs 1 if \textbf{at least one} input is 1. (Logical disjunction; symbol $\lor$ or +)

\noindent NOT Gate = Logic gate that outputs the logical \textbf{negation} of its input (outputs 1 if input is 0, and vice versa). (Inversion; symbol $\lnot$ or ! or ')

\noindent Logic 1 (HIGH) = The higher logic level in binary (e.g., True, "1", high voltage level) representing boolean true.

\noindent Logic 0 (LOW)  = The lower logic level (e.g., False, "0", low voltage level) representing boolean false.

\noindent Truth Table = A table listing all possible combinations of input values and the corresponding output for a logic function. Used to define the behavior of the logic circuit exhaustively.

\noindent Karnaugh Map (K-Map) = A graphical method for simplifying Boolean expressions by organizing truth table values into a grid, allowing common terms to be combined. It helps reduce logic expressions to minimal form.

\noindent Sum-of-Products (SOP) = A canonical Boolean form where the expression is a OR (sum) of multiple AND (product) terms. Each product term corresponds to a combination of inputs that yields output 1.

\noindent Boolean Expression = An algebraic expression composed of boolean variables and logic operators (AND, OR, NOT, etc.) representing a logic function. E.g., F = A$\cdot$B + $\lnot$C.

\noindent Node (Intermediate) = A connection or intermediate signal in the circuit (output of a gate that is not the final output). Often labeled (e.g., X, Y) and has its own Boolean equation in the context of the circuit.

\noindent Node Equation = The Boolean expression for an intermediate node's value in terms of the circuit's input variables (and possibly other node values). E.g., X = A$\cdot$B defines node X as the AND of A and B.

\noindent Propagation Delay = The time interval between an input change and the resulting output change in a logic circuit. Every logic gate has a finite switching speed, so signals take time to propagate through the circuit (see also Gate Delay).

\noindent Signal Propagation = The movement of logic level changes through the circuit's network of gates and connections, from inputs to intermediate nodes to outputs.

\noindent Verification Checklist = A list of steps or criteria used to verify the correctness of each part of the design. Each item is marked with a $\checkmark$ if passed or $\times$ if an issue is found, accompanied by comments explaining the result.

\noindent Cross-Referencing = The practice of linking related sections or terms in the document for easy navigation. For example, referring to the Truth Table section when discussing K-Map, or pointing to Glossary definitions for specific terms.

\noindent $\checkmark$ (Check mark) = Indicates a successful verification or a condition met. In the template, replace this symbol in a checkbox once the verification step is confirmed.

\noindent $\times$ (Cross mark) = Indicates a failed verification or a condition not met. It denotes an issue that needs attention or correction, explained in the comments.

\vspace{0.5em}

\noindent [SIMULATION\_AND\_RESULTS]

\noindent simulation\_tool=Tina

\noindent test\_duration=100ns

\noindent input\_stimulus="""

\noindent A: 0->1 at 20ns, 1->0 at 60ns

\noindent B: 0->1 at 40ns, 1->0 at 80ns

\noindent """

\vspace{0.5em}

\noindent [WAVEFORM\_SETTINGS]

\noindent display\_configuration="""

\noindent Signal Colors:

\noindent - A: Blue (RGB: 0,0,255)

\noindent - B: Red (RGB: 255,0,0)

\noindent - Sum: Green (RGB: 0,255,0)

\noindent - Carry: Yellow (RGB: 255,255,0)

\vspace{0.5em}

\noindent Trace Settings:

\noindent - Data Trace Width: 2px

\noindent - Background: White (RGB: 255,255,255)

\noindent - Grid: Light Gray (RGB: 200,200,200)

\noindent """

\vspace{0.5em}

\noindent [SIMULATION\_RESULTS]

\noindent waveform\_analysis="""
\begin{tabular}{c|c|c|c|c|l}
Time     & A & B & Sum & Carry & Analysis \\
0-20ns   & 0 & 0 &  0  &   0   & Initial state \\
20-40ns  & 1 & 0 &  1  &   0   & A transition \\
40-60ns  & 1 & 1 &  0  &   1   & B transition, carry generated \\
60-80ns  & 0 & 1 &  1  &   0   & A transition \\
80-100ns & 0 & 0 &  0  &   0   & Return to initial state
\end{tabular}

\noindent """

\vspace{0.5em}

\noindent [VISUALIZATION\_TIPS]

\noindent waveform\_optimization="""

\noindent 1. Right-click signal name above graph to modify colors

\noindent 2. Access Tools-->Settings-->Waveforms for trace width

\noindent 3. Modify background via Tools-->Color preferences-->Waveform

\noindent 4. Set RGB values to 255 for maximum visibility

\noindent """

\vspace{0.5em}

\noindent [TROUBLESHOOTING\_NOTES]

\noindent initial\_attempts="""

\noindent 1. First attempt: Four inputs (redundant)

\noindent    - Issue: Overcomplicated design

\noindent    - Resolution: Simplified to two inputs

\vspace{0.5em}

\noindent 2. Carry implementation:

\noindent    - Initial mistake: Added C input for carry

\noindent    - Correction: Generated carry from A,B inputs

\noindent    - Learning: Document mistakes to prevent future repetition

\noindent """

\vspace{0.5em}

\noindent [DESIGN\_ITERATIONS]

\noindent version\_history="""

\noindent v1: Basic design with redundant inputs

\noindent v2: Removed redundant inputs

\noindent v3: Corrected carry generation

\noindent v4: Final optimized design with simulation verification

\noindent """

\noindent Let's combine and streamline these templates into a more focused structure for

\noindent your half-adder design documentation:

\vspace{0.5em}

\noindent [DOCUMENT\_METADATA]

\noindent title=Half Adder Design Documentation

\noindent version=1.0

\noindent date=2024-02-XX

\noindent author=Student\_Name

\noindent course=Digital\_Design\_Lab\_Two

\vspace{0.5em}

\noindent [NOTATION\_AND\_SYMBOLS]

\noindent gate\_symbols="""

\noindent NAND = $\overline{AB}$

\noindent AND = $\cdot$

\noindent OR = +

\noindent XOR = $\oplus$

\noindent NOT = $\lnot$

\noindent """

\vspace{0.5em}

\noindent circuit\_symbols="""

\noindent Wire = \textemdash

\noindent Junction = $\vdash$

\noindent Turn =  $\lfloor$ $\lceil$

\noindent Signal Flow = $\rightarrow$

\noindent Inversion = $\circ$

\noindent """

\vspace{0.5em}

\noindent logic\_levels="""

\noindent HIGH = 1 (True)

\noindent LOW = 0 (False)

\noindent """

\vspace{0.5em}

\noindent [DESIGN\_SPECIFICATIONS]

\noindent block\_type=Half\_Adder

\noindent inputs=A,B

\noindent outputs=Sum,Carry

\noindent constraints=carry\_in=fixed\_zero

\noindent implementation=NAND\_Based

\vspace{0.5em}

\noindent [IMPLEMENTATION\_DETAILS]

\noindent circuit\_diagram="""

\noindent A \textemdash$\vdash$\textemdash$\rightarrow$ [NAND1] \textemdash$\vdash$\textemdash$\rightarrow$ [NAND3] \textemdash$\rightarrow$ [OR] \textemdash$\rightarrow$ Sum

\noindent    $|$              $|$

\noindent B \textemdash$\dashv$\textemdash$\rightarrow$ [NAND2] \textemdash$\rfloor$

\noindent    $|$

\noindent    $\lfloor$\textemdash$\rightarrow$ [NAND4] \textemdash$\rightarrow$ Carry

\noindent """

\vspace{0.5em}

\noindent truth\_table="""
\begin{tabular}{|c|c|c|c|}
\hline
A & B & Sum & Carry \\
\hline
0 & 0 & 0   & 0     \\
0 & 1 & 1   & 0     \\
1 & 0 & 1   & 0     \\
1 & 1 & 0   & 1     \\
\hline
\end{tabular}

\noindent """

\vspace{0.5em}

\noindent [SIMULATION\_SETUP]

\noindent tool=Tina

\noindent duration=100ns

\noindent input\_stimulus="""

\noindent A: 0->1 at 20ns, 1->0 at 60ns

\noindent B: 0->1 at 40ns, 1->0 at 80ns

\noindent """

\vspace{0.5em}

\noindent [WAVEFORM\_CONFIGURATION]

\noindent display\_settings="""

\noindent Colors:

\noindent - A: Blue (0,0,255)

\noindent - B: Red (255,0,0)

\noindent - Sum: Green (0,255,0)

\noindent - Carry: Yellow (255,255,0)

\vspace{0.5em}

\noindent Trace Width: 2px

\noindent Background: White

\noindent Grid: Light Gray

\noindent """

\vspace{0.5em}

\noindent [RESULTS\_AND\_ANALYSIS]

\noindent timing\_analysis="""
\begin{tabular}{c|c|c|c|c|l}
Time     & A & B & Sum & Carry & Analysis \\
0-20ns   & 0 & 0 &  0  &   0   & Initial state \\
20-40ns  & 1 & 0 &  1  &   0   & A transition \\
40-60ns  & 1 & 1 &  0  &   1   & B transition, carry generated \\
60-80ns  & 0 & 1 &  1  &   0   & A transition \\
80-100ns & 0 & 0 &  0  &   0   & Return to initial state
\end{tabular}
\noindent """

\vspace{0.5em}

\noindent [DESIGN\_HISTORY]

\noindent iterations="""

\noindent v1: Initial design (redundant inputs)

\noindent v2: Simplified inputs

\noindent v3: Corrected carry generation

\noindent v4: Final verified design

\noindent """

\vspace{0.5em}

\noindent [VISUALIZATION\_GUIDELINES]

\noindent setup\_steps="""

\noindent 1. Right-click signal names for color options

\noindent 2. Tools-->Settings-->Waveforms for trace width

\noindent 3. Tools-->Color preferences for background

\noindent 4. Maximize RGB values (255) for clarity

\noindent """

\vspace{0.5em}

\noindent [VERIFICATION\_CHECKLIST]

\noindent $\checkmark$ truth\_table\_verified

\noindent $\checkmark$ timing\_requirements\_met

\noindent $\checkmark$ carry\_generation\_correct

\noindent $\checkmark$ waveform\_visualization\_optimized
\end{document}